\documentclass[12pt]{article}

\title{\textbf{Generative Adversarial Networks for Text Generation}}
\author{Sven Vaupel}
\date{xx.xx.2017}

\usepackage{amsmath}
\usepackage{amssymb}
\usepackage{amstext}
\usepackage{amsfonts}
\usepackage{mathrsfs}
\usepackage{import}
\usepackage{algorithm}
\usepackage{cleveref}

% TODO: Plugins anpassen!

\crefname{algorithm}{Algorithm}{Algorithms}
\crefname{equation}{Equation}{Equations}

\begin{document}

\maketitle
\setlength{\parindent}{0cm}

\section{Overview}

Giving a basic view over what is planned. Especially what I plan to describe before I tackle GANs!

\begin{enumerate}
  \item Notation ("copy" from \cite{2})

  \item Neural Networks \\
    What are Neural Networks and how do they work. A brief introduction.

  \begin{enumerate}

    \item Use cases \\
      Giving a quick overview of Regression and Classification using simple examples.

    \item Important terms \\
      Giving a quick overview over the terminology used specifically in machine learning

      \begin{enumerate}

        \item Hyperparameter \\
          The Hyperparameters' values are set prior to the actual training process.

        \item Stochastic gradient

        \item Over- / Underfitting

      \end{enumerate}

    \item Important concepts \\
      Concepts which are important for the explanation of later topics

    \begin{enumerate}

      \item Backpropagation \\
        Why is it Important and what are the requirements.

      \item Minibatch Training

      \item Dropout

      \item Latent space? \\
        Could be of interest for some of the papers.
        What Information can be drawn from it. Showing some examples.

    \end{enumerate}

  \end{enumerate}

  \item GAN
\end{enumerate}


\import{sections/}{gan_section/gan}

% examples
% \emph{emph}\footnote{footnote}
% $\rightarrow$

\bibliography{bibliography.bib}{}
\bibliographystyle{IEEEtran}
\end{document}
